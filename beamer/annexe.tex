\addtocounter{framenumber}{-1}
\setbeamertemplate{footline}{}
\subsection*{Démonstrations}


\label{demo:congdecarre}
\begin{frame}
    \addtocounter{framenumber}{-1}
    \begin{block}{Proposition}
    Soient $b\in \mathbb N, (x_i)_{i\in \llbracket 1, b+1\rrbracket}\in \mathbb N^{b+1}$
    et $(v_i)_{i\in \llbracket 1, b+1\rrbracket}\in \mathbb F_2^b$ les vecteurs valuations de $x_i^2 \pmod N$ pour $i \in \llbracket 1,b+1 \rrbracket$
    et finalement $(\lambda_i)_{i \in \llbracket 1,b+1 \rrbracket}\in \mathbb \{0,1\}^{b+1}$ tels que,
    \[
    \sum_{i=1}^{b+1} \lambda_iv_i = 0_{\mathbb F_2^b} = (2\alpha_1, \dots, 2\alpha_b)
    \]
    \newline
    On pose $y = \prod_{j=1}^b p_j^{\alpha_j}$ et $x = \prod_{j=1}^{b+1}x_j^{\lambda_j}$, alors $x ^2 \equiv y ^2 \pmod N$
    \end{block}
\end{frame}

\begin{frame}
    \addtocounter{framenumber}{-1}
    \begin{block}{Démonstration}
        \begin{align*}
            x^2 = (\prod_{i=1}^{b+1} x_i^2)^{\lambda_i} &\equiv \prod_{i=1}^{b+1} \prod_{j=1}^b p_j^{\lambda_iv_i^{(j)}} \pmod N\\
            &\equiv \prod_{j=1}^b \prod_{i=1}^{b+1} p_j^{\lambda_iv_i^{(j)}} \pmod N\\
            &\equiv \prod_{j=1}^b p_j^{\sum_{i =1}^{b+1}\lambda_iv_i^{(j)}} \pmod N\\
            &\equiv (\prod_{j=1}^b p_j^{\alpha_j})^2 \pmod N \qquad\quad (\text{déf de } \alpha_j) \\ % cursed but works
            &\equiv y^2 \pmod N
        \end{align*}
    \end{block}
\end{frame}

\label{demo:qx}
\begin{frame}
    \addtocounter{framenumber}{-1}
    \begin{block}{Proposition}
        Si $Q = (\lfloor\sqrt N\rfloor + X)^2 - N$, alors $p\mid Q(x) \implies \forall k\in \mathbb N,  p\mid Q(x+kp)$
    \end{block}
    \begin{block}{Démonstration}
    En effet, supposons $p\mid Q(x)$, on a:
    \begin{align*}
        Q(x+kp) &= (\lfloor\sqrt N\rfloor + x+kp)^2 - N \\
                &= Q(x) + 2kp(\lfloor\sqrt N\rfloor+ x) + k^2p^2 \\
                &= Q(x) + p\times(2k(\lfloor\sqrt N\rfloor + x) + k^2p)
    \end{align*}
    d'où $p\mid Q(x+kp)$
    \end{block}
\end{frame}

{
\UseRawInputEncoding

%Make plain
\setbeamertemplate{headline}{}
\setbeamertemplate{navigation symbols}{}
%

%Set code colours and style
\definecolor{keywordsColour}{RGB}{0, 0, 255}
\definecolor{stringsColour}{RGB}{162, 13, 13}
\definecolor{commentsColour}{RGB}{0, 127, 0}
\definecolor{identifierColour}{RGB}{0, 0, 0}

\lstdefinestyle{fullStyle}{
    breaklines=true,
    columns=fullflexible,
    language=C,
    basicstyle=\tiny,
    keywordstyle=\bfseries\color{keywordsColour},
    commentstyle=\itshape\color{commentsColour},
    identifierstyle=\color{identifierColour},
    stringstyle=\color{stringsColour},
}
%

% Import code command
\newcommand{\importcode}[1]{
    \begin{flushright}\textit{#1}\end{flushright}
    \begin{multicols}{2}
        \lstinputlisting[style=fullStyle]{#1}
    \end{multicols}
    \pagebreak
}
%

\subsection*{Codes}

\importcode{../c/vector.h}
\importcode{../c/vector.c}
\importcode{../c/tonellishanks.h}
\importcode{../c/tonellishanks.c}
\importcode{../c/system.h}
\importcode{../c/system.c}
\importcode{../c/parse\_input.h}
\importcode{../c/parse\_input.c}
\importcode{../c/list\_matrix\_utils.h}
\importcode{../c/list\_matrix\_utils.c}
\importcode{../c/factorbase.h}
\importcode{../c/factorbase.c}
\importcode{../c/main.c}

\importcode{../c/dixon/dixon.h}
\importcode{../c/dixon/dixon.c}

\importcode{../c/qsieve/qsieve.h}
\importcode{../c/qsieve/qsieve.c}

\importcode{../c/mpqs/common\_mpqs.h}
\importcode{../c/mpqs/common\_mpqs.c}
\importcode{../c/mpqs/polynomial.h}
\importcode{../c/mpqs/polynomial.c}
\importcode{../c/mpqs/mpqs.h}
\importcode{../c/mpqs/mpqs.c}
\importcode{../c/mpqs/parallel\_mpqs.h}
\importcode{../c/mpqs/parallel\_mpqs.c}
}